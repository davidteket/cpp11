\documentclass{article}

\begin{document}
\section*{My top 10 list of helpful design and programming rules}
\textbf{Design:}
 \begin{enumerate}
  \item  Represent ideas (concepts) directly in code, for example, as a function, a class, or an enumeration.
  \item  Represent independent ideas separately in code, for example, avoid mutual dependencies among classes.
  \item  Use libraries, especially the standard library, rather than trying to build everything from scratch.
  \item  Represent independent ideas independently in code.
  \item  Keep information local (e.g., avoid global variables, minimize the use of pointers).
  \item  Aim for your code to be both elegant and efficient.
  \item  Don’t overabstract.
  \item  Represent relationships among ideas directly in code, for example, through parameterization or a class hierarchy.
  \item  Prefer solutions that can be statically checked.
 \end{enumerate}
 
\textbf{Programming:}
\begin{enumerate}
 \item Use constructors to establish invariants.
 \item Use constructor/destructor pairs to simplify resource management (RAII).
 \item Avoid ‘‘naked’’ \texttt{new} and \texttt{delete}.
 \item Use containers and algorithms rather than built-in arrays and ad hoc code.
 \item Prefer standard-library facilities to locally developed code.
 \item Use exceptions, rather than error codes, to report errors that cannot be handled locally.
 \item Use move semantics to avoid copying large objects.
 \item Use \texttt{unique\_ptr} to reference objects of polymorphic type.
 \item Use \texttt{shared\_ptr} to reference shared objects, that is, objects without a single owner that is responsible for their destruction.
 \item Use templates to maintain static type safety (eliminate casts) and avoid unnecessary use of class hierarchies.
\end{enumerate}


\end{document}
