\documentclass{article}

\begin{document}
 \section*{Analyzing a non-trivial algorithm}
 \texttt{void send(int* to, int* from, int count)\\
    // Duff ’s device. Helpful comment deliberately deleted.\\
    \{\\
      n = (count+7)/8;\\
      switch (count\%8) \{\\
      case 0: do \{ *to++ = *from++;\\
      case 7: *to++ = *from++;\\
      case 6: *to++ = *from++;\\
      case 5: *to++ = *from++;\\
      case 4: *to++ = *from++;\\
      case 3: *to++ = *from++;\\
      case 2: *to++ = *from++;\\
      case 1: *to++ = *from++;\\
        \} while (−−n>0);\\
      \}\\
    \}}
    \\
    \\
    \begin{itemize}
     \item The argument \texttt{count} implies that the source (\texttt{from}) and the destination (\texttt{to}) are arrays of integers,
     and \texttt{count} indicates the number of integers that needs to be copied (\texttt{send}) to the destinatition (\texttt{to}) array.
     \item \texttt{(count + 7)} resembles to a positive offset with \textsl{one byte} thus dividing it by $8$ (one byte) results in the number of bytes, \\
     so \texttt{n is the number of bytes}.
     \item \texttt{switch (count\%8)} if count was 10 then it would result to 2. This way n would also result to 2 \\
     ($\Rightarrow (16 + 7) / 8 = 2)$)\\
     ($\Rightarrow 16\ mod\ 8 = 0$)
     \item The do-while loop operates \textsl{only if count is the multiply of 8} and while n is not equal to 0. \\
     Inside the loop as soon as a corresponding case is found ($0$ to start the loop), all the remaining cases are being executed
     beginning with (\texttt{case 7: *to++ = *from++;} to (\texttt{case 1: *to++ = *from++;}.
    \end{itemize}
    
  \section*{Conclusion}
  The algorithm above \textsl{sends messages} represented as integers in \textsl{count} packets from destination \textsl{from} to destination \textsl{to}.
  Count corresponds to the number of packets, whereas one packet must be the multiply of 8. 
  \\
  \\
  A packet is being sent in every iteration.

 
\end{document}
